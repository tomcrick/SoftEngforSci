\documentclass[a4paper,11pt]{article}
\usepackage[top=2cm,bottom=2cm,left=2cm,right=2cm,asymmetric]{geometry}
\usepackage{url}
\usepackage{paralist}
\usepackage{authblk}
\usepackage[colorlinks=true,hyperfootnotes=true]{hyperref}

\title{Chapter Proposal for Software Engineering for Science:\\
  \emph{Best Practices for Sustainable Scientific Software Development}}

% names
\author[1]{Tom Crick}
\author[2]{Benjamin A. Hall}
\author[3]{Samin Ishtiaq}
% affiliation
\affil[1]{Department of Computing \& Information Systems, Cardiff
  Metropolitan University, UK}
\affil[2]{MRC Cancer Unit, University of Cambridge, UK}
\affil[3]{Microsoft Research Cambridge, UK}
% emails
\affil[1]{\protect\url{tcrick@cardiffmet.ac.uk}}
\affil[2]{\protect\url{bh418@mrc-cu.cam.ac.uk}}
\affil[3]{\protect\url{samin.ishtiaq@microsoft.com}}

\renewcommand\Authands{ and }
\def\UrlBreaks{\do\/\do-}

\date{ }

\begin{document}
\maketitle

\vspace{-1cm}
\begin{abstract}
Research software development represents a borderland in the explosion
of code across the sciences, engineering and other increasingly
computational disciplines. It combines problems and people in a unique
way; the researchers involved are not programmers, and do not
typically have the skill sets and broader awareness of standard
approaches to ensure code reliability.  Fixing this is however not
just issue of education or culture; in contrast to the shifting
constraints and requirements an industry programmer may face, the
specifications for scientific software may simply not exist. People
will approach problems from multiple angles, effectively panning for
research gold, unsure of their tools and reliant on their own
judgement.  The fundamental problem --- not knowing what the ``right''
answer should be, whilst attempting to make rigorous reproducible
discoveries --- requires a modified approach. In this chaper, we
highlight the pressures and problems of working in this rarefied
environment, encapsulating lessons which are valuable for both
scientific and software cultures.
\end{abstract}

\section*{Outline}

In this chapter, we will highlight and discuss not only the issues
which may arise in interdisciplinary computational research but also
to draw out the characteristics of research software development and
how it is distinguished from conventional software engineering. We
will present the various issues which arise that are unique or
abundant in research software and how we have encountered and
(sometimes) successfully overcome them. Through a structured
narrative, we will draw from our collective wide range of experiences,
contextualised and illustrated by two key scientific software projects
which we have been involved with: the
{\emph{BioModelAnalyzer}}\footnote{\url{http://biomodelanalyzer.research.microsoft.com/}}~\cite{benque-et-al:2012};
and the more recent, related {\emph{Cellular Dynamics
Engine}}~\cite{hall-et-al:2015}. We intend to cover specifically:\\

\begin{compactitem}
\item Real-world research context:
\begin{compactitem}
\item Balancing scientific requirements (such as implementing alternative 
	approaches rapidly) against common engineering needs of code optimisation, 
	readability, etc; 
\item User-driven discovery: the unexpected identification of 
	scientific/algorithmic issues;
\item The linking of {\em in vivo} and {\em in silico} work flows;
\item Socio-cultural and cyberinfrastructural issues of embedding research software
  sustainability and reproducibility into workflows;
\end{compactitem}
\item Software engineering challenges:
\begin{compactitem}
\item The challenges of testing in an environment where behaviours 
	are not known \emph{a priori}; 
\item Issues encountered when separating models and code; 
\item Issues encountered in maintaining a scientific codebase in a diverse, 
cross-disciplinary project; 
\item Issues encountered in maintaining and extending a model database
  in a diverse, cross-disciplinary project;
\end{compactitem}
\item Recommendations and sharing of best practice:
\begin{compactitem}
\item User interface development to help in cross-disciplinary work;
\item Practices we have used that can and should be more widely adopted in
  scientific environments, for example: version control, unit testing,
  continuous integration, code review, robust documentation, etc. 
\item A roadmap for researchers of how best to integrate and adopt these practices.\newline
\end{compactitem}
\end{compactitem}

% http://se4science.org/BookChapters.htm
We envisage this chapter sitting in {\emph{Section I: Examples of the
Application of Traditional Software Engineering Practices to
Science}}, as it will describe a number of generalisable experiences
from our background of applying modern software engineering practices
to the development of scientific software. The topics listed above
will build on each other through the narrative, illustrated and
contextualised throughout the chapter with integrated case studies and
real world examples.

While we recognise that we will present a number of recommendations
for improving scientific software that may already be known
individually, we will provide a clear narrative and compelling
real-world examples to highlight where we have been successful (and
perhaps more usefully, unsuccessful), as well as addressing some of
the wider cultural issues within the scientific research community on
how these problems can be made more tractable for scientific software
developers and how these practices could be more widely adopted.

\section*{Authors}

The three authors have wide experience of scientific software
engineering, from real-world tool development across a number of
application domains, through to associated policy. They have
previously published work in this space, focusing on reproducibility
and sharing of scientific software
models~\cite{crick-et-al_wssspe2,crick-et-al_recomp2014,crick-et-al_jors},
as well as proposing e-infrastructure and workflow models for
encouraging and supporting reproducibility in computational science
and engineering\footnote{\url{http://arxiv.org/a/crick_t_1.html}}.

Dr Tom Crick\footnote{\url{http://drtomcrick.com}} is a Senior
Lecturer in Computing Science at Cardiff Metropolitan University,
having completed his PhD and post-doctoral research at the University
of Bath. His research interests cut across data and
computational-intensive domains: data science, knowledge
representation and reasoning, optimisation, social network analysis
and high performance computing. He is the Nesta Data Science Fellow, a
2014 Fellow of the Software Sustainability Institute (EPSRC) and a
member of {\emph{HiPEAC}}, the European FP7 Network of Excellence on
High Performance and Embedded Architecture and Compilation.

Dr Benjamin
A. Hall\footnote{\url{http://www.mrc-cu.cam.ac.uk/hall.html}} is a
Royal Society University Research Fellow, developing hybrid and formal
models of carcinogenesis and biological signalling at the MRC Cancer
Unit, University of Cambridge. He previously worked at Microsoft
Research Cambridge (on {\emph{BioModelAnalyzer}} with Jasmine Fisher), UCL and
the University of Oxford. As part of his role at Oxford, he was one of
two Apple Laureates, awarded by Apple and the Oxford Supercomputing
Centre for the project {\emph{A biomolecular simulation
pipeline}}. Benjamin has an MBiochem and DPhil from the University of
Oxford.

Dr Samin
Ishtiaq\footnote{\url{http://research.microsoft.com/en-us/people/sishtiaq/}}
is Principal Engineer in the Programming Principles and Tools group at
Microsoft Research Cambridge. He currently works on the {\emph{SLAyer}}
(Separation Logic-based memory safety for C programs), {\emph{TERMINATOR}}
(program termination) and {\emph{BioModelAnalyzer}} projects. Samin joined MSR in April
2008. Before that (2000-2008), he worked in CPU modelling and
verification at ARM, helping to tape-out the Cortex A8, Cortex M3 and
SC300 processors, and the AMBA bus protocol checker. Samin has an MEng
from Imperial and a PhD in dependent type theory from Queen Mary.

% bib
\bibliographystyle{unsrt}
\bibliography{softengforsci}

\end{document}
