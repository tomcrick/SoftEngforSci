\documentclass[a4paper,11pt]{article}
\usepackage[top=2cm,bottom=2cm,left=2cm,right=2cm,asymmetric]{geometry}
\usepackage{url}
\usepackage{paralist}
\usepackage{authblk}
\usepackage[colorlinks=true,hyperfootnotes=true]{hyperref}

\title{Chapter Proposal for Software Engineering for Science}

% names
\author[1]{Tom Crick}
\author[2]{Benjamin A. Hall}
\author[3]{Samin Ishtiaq}
% affiliation
\affil[1]{Department of Computing \& Information Systems, Cardiff
  Metropolitan University, UK}
\affil[2]{MRC Cancer Unit, University of Cambridge, UK}
\affil[3]{Microsoft Research Cambridge, UK}
% emails
\affil[1]{\protect\url{tcrick@cardiffmet.ac.uk}}
\affil[2]{\protect\url{bh418@mrc-cu.cam.ac.uk}}
\affil[3]{\protect\url{samin.ishtiaq@microsoft.com}}

\renewcommand\Authands{ and }
\def\UrlBreaks{\do\/\do-}

\date{ }

\begin{document}
\maketitle


\section*{Authors and qualifications}

Dr Tom Crick\footnote{\url{http://drtomcrick.com}} is a Senior
Lecturer in Computing Science at Cardiff Metropolitan University,
having completed his PhD and post-doctoral research at the University
of Bath. His research interests cut across computational science:
knowledge representation and reasoning, big data, social network
analysis, optimisation and high performance computing. He is the
Nesta Data Science Fellow, a 2014 Fellow of the Software
Sustainability Institute (EPSRC) and a member of {\emph{HiPEAC}}, the
European FP7 Network of Excellence on High Performance and Embedded
Architecture and Compilation.

Dr Benjamin A. Hall\footnote{\url{http://www.mrc-cu.cam.ac.uk/hall.html}} is a
Royal Society University Research Fellow, developing hybrid and formal
models of carcinogenesis and biological signalling at the MRC Cancer
Unit, University of Cambridge. He previously worked at Microsoft
Research Cambridge (on BMA with Jasmine Fisher), UCL and the
University of Oxford. As part of his role at Oxford, he was one of two
Apple Laureates, awarded by Apple and the Oxford Supercomputing Centre
for the project {\emph{A biomolecular simulation pipeline}}. Benjamin
has an MBiochem and DPhil from the University of Oxford.

Dr Samin Ishtiaq\footnote{\url{http://research.microsoft.com/en-us/people/sishtiaq/}}
is Principal Engineer in the Programming Principles and Tools group at
Microsoft Research Cambridge (UK). He currently works on the SLAyer
(Separation Logic-based memory safety for C programs), TERMINATOR
(program termination) and BMA projects. Samin joined MSR in April
2008. Before that (2000-2008), he worked in CPU modelling and
verification at ARM, helping to tape-out the Cortex A8, Cortex M3 and
SC300 processors, and the AMBA bus protocol checker. Samin has an MEng
from Imperial and a PhD in dependent type theory from Queen Mary.

\section*{Chapter title: {\emph{TBC}}}

\section*{Section of the book}

% http://se4science.org/BookChapters.htm
We envisage this chapter sitting in {\emph{Section I: Examples of the
Application of Traditional Software Engineering Practices to
Science}}, as it will describe generalisable experiences with applying
traditional software engineering practices to the development of
scientific software particularly, illustrated by an appropriate case
studies and real world examples. Our intent here is to discuss the 
not only the issues which may arise in interdisciplinary work (which
have been covered in detail elsewhere) but rather to draw out the
characteristics of research scientific software development which 
distinguish it from conventional software engineering. In it we hope
to highlight the issues which arise that are unique to research software
and how they we have encountered and (sometimes) successfully overcome 
them. In this chapter we will draw from our wide range of experiences, but
we will focus on two software projects which we have been involved with-
the BioModelAnalyzer (Benque et al) and the more recent, related Cellular 
Dynamics Engine (Hall et al, Biophysical Journal 2015). We intend to cover 
specifically:

\begin{itemize}
\item the challenges of testing in an environment where behaviours 
	are not known \emph{a priori}
\item balancing scientific requirements (such as implementing alternative 
	approaches rapidly) against common needs code optimsation, 
	readability etc
\item user driven discovery- the unexpected identifaction of 
	scientific/algorithmic issues
\item issues encountered when separating models and code
\item user interface development to bridge disciplines
\item chapters describing practices that are being adopted in
  scientific environments, e.g. version control or unit testing;
\item chapters describing practices that have potential benefit to
  science but currently lack adoption, sufficient evidence, or case
  studies, e.g. Continuous Integration, code review, documentation;
\end{itemize}

\section*{Abstract}

Abstract here...

\section*{Detailed outline}

\section*{Justification for inclusion}

Cite the previous
publications~\cite{crick-et-al_wssspe2,crick-et-al_recomp2014,crick-et-al_jors,crick-et-al_cse2015}?

Also refer to other arXiv papers\footnote{\url{http://arxiv.org/a/crick_t_1.html}}?


% bib
\bibliographystyle{unsrt}
\bibliography{softengforsci}

\end{document}
